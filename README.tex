\documentclass[a4paper,11pt]{article}
\usepackage{standalone}
\standalonetrue
\usepackage{natbib}
\usepackage{html}
\usepackage{latexsym}
\usepackage{makeidx}

\usepackage{color}
\usepackage{hyperref}
\hypersetup{
    colorlinks=true,
    linkcolor=blue,
    urlcolor=blue,
    linktoc=all
}

\newcommand{\dvdate}{2020/09/07}
\newcommand{\dvversion}{Alpha 0.0}

\usepackage{polyglossia}
\setdefaultlanguage{english}
\setotherlanguage{pali}
\enablehyphenation

\pghyphenation{english}{
   bhik-khus
   dham-ma
   san-gha
}

%Hyperlinks:
\usepackage{hyperref}
\hypersetup{pdftex,colorlinks=true,allcolors=blue,pdfpagemode=UseOutlines}
\usepackage{hypcap}
%%%%%%%%%%%%%%%
\usepackage[final]{microtype}
\usepackage{changepage}
\usepackage{tocloft}
\usepackage{luatexbase}
\usepackage{currfile}
\setmainfont{Charis SIL}
\newfontfamily\palifont{DejaVu Serif}

\begin{document}

\ifstandalone
\title{Saddhamma Translation}
\author{Ebo Croffie}

\date{\dvdate \qquad \dvversion\\
\footnotesize (\textsc{pdf} file generated on \today)}

\maketitle

\fi

\begin{fussy}
\large\textenglish{
\paragraph{1.} The Saddhamma Translation (translate-saddhamma.git) project is a translation of selected
suttas by Ebo Croffie for a weekly \htmladdnormallink{Saddhamma Study Group}
{mailto:saddhammastudygroup@gmail.com} readings and discussions.

\paragraph{2.} These translations are designed to preserve the right {\em meaning and
phrasing (sātthaṃ sabyañjanaṃ)} as much as possible, in hopes that they
will help reveal the life of purity that is utterly perfect and pure to the
study group participants who practice accordingly, as expected of the Dhamma
that is beautiful in the beginning, beautiful in the middle and beautiful in the end, with
the {\em right meaning and phrasing}. However, the reader should keep in mind
that even with these best efforts at preserving translation fidelity, some loss
of phrasing and consequently, loss in meaning is unavoidable due to English
language limitations. The reader is therefore advised to become familiar enough
with the original Pali texts to enable cross-checking translations with the Pali
to avoid being misled.

\paragraph{3.} The website framework (designed by the
\htmladdnormallink{The Buddha's Words} {mailto:buddhaswords2@gmail.com} team)
used for hosting these translations makes such cross-checking easily
accessible with side-by-side Pali and English renderings. It also features
built-in Pali dictionary and Pali/English search capabilities for referencing
other suttas with similar phrasing.
}
\end{fussy}
\end{document}
